\documentclass{article}
\usepackage[final]{nips_2017}
\usepackage[utf8]{inputenc} % allow utf-8 input
\usepackage[T1]{fontenc}    % use 8-bit T1 fonts
\usepackage{hyperref}       % hyperlinks
\usepackage{url}            % simple URL typesetting
\usepackage{booktabs}       % professional-quality tables
\usepackage{amsfonts}       % blackboard math symbols
\usepackage{nicefrac}       % compact symbols for 1/2, etc.
\usepackage{microtype}      % microtypography
\usepackage{graphicx}
\title{Spoken Command Recognition}

\author{
  Thomas Karpati \\
  Department of Computer Science\\
  Stanford University\\
  \texttt{tkarpati@stanford.edu} \\
  %% examples of more authors
  %% \And
  %% Coauthor \\
  %% Affiliation \\
  %% Address \\
  %% \texttt{email} \\
  %% \AND
  %% Coauthor \\
  %% Affiliation \\
  %% Address \\
  %% \texttt{email} \\
  %% \And
  %% Coauthor \\
  %% Affiliation \\
  %% Address \\
  %% \texttt{email} \\
  %% \And
  %% Coauthor \\
  %% Affiliation \\
  %% Address \\
  %% \texttt{email} \\
}

\begin{document}
% \nipsfinalcopy is no longer used

\begin{center}
\includegraphics[width=3cm, height=0.7cm]{CS230}
\end{center}

\maketitle

\begin{abstract}
  Ineractions with agents in the world has been increasing using voice
  commands. This allows users to interact without the use of a
  terminal input such as when they are not in close proximity, are
  otherwise physically occupied, or enable users such as children who
  are illiterate and cannot use standard computing interaces which
  would require the ability to read. In the simple case such an agent
  would need to be able to recognize basic commands to perform
  tasks. One of the preprocessing steps which is common for speech
  recognition is the use of the spectrograph. This operation converts
  the single dimension of an audio saple to a two dimensional
  representation of the audio waveform through a basis
  transformation. This representation can be processed as a two
  dimensional image to extract features before passing this to a
  sequential model. 
\end{abstract}

\section{Introduction}
Speech recognition is allowing more accessible interactions with
agents. Deep learning systems have allowed for improved speech
recognition accuracy through sequential models. Often the inputs to
these sequential models is the spectrogram representation of an audio
waveform. The spectrogram is a representation which converts the time
domain audio waveform into a multidimensional basis space for segments
of time within a sliding window. The result of this is a
multidimensional representation of the audio waveform during a
specified period of time. As these are accumulated over time, this can
be seen as a two dimensional representation of the audio waveform. Our
alogrithm explores applying two dimmensional network architectures to
a traditional sequential model for speech recognition processing. 




Explain the problem and why it is important. Discuss your motivation for pursuing this
problem. Give some background if necessary. Clearly state what the input and output
is. Be very explicit: “The input to our algorithm is an {image, amplitude, patient age,
rainfall measurements, grayscale video, etc.}. We then use a {SVM, neural network, linear
regression, etc.} to output a predicted {age, stock price, cancer type, music genre, etc.}.”
This is very important since different teams have different inputs/outputs spanning different
application domains. Being explicit about this makes it easier for readers. If you are using
your project for multiple classes, add a paragraph explaining which components of the
project were used for each class.

\section{Related work}
You should find existing papers, group them into categories based on their approaches,
and discuss their strengths and weaknesses, as well as how they are similar to and differ
from your work. In your opinion, which approaches were clever/good? What is the stateof-the-art?
Do most people perform the task by hand? You should aim to have at least
5 references in the related work. Include previous attempts by others at your problem,
previous technical methods, or previous learning algorithms. Google Scholar is very useful
for this: https://scholar.google.com/ (you can click “cite” and it generates MLA, APA,
BibTeX, etc.)

\section{Dataset and Features}
To evaluate models for spoken command recognition, the dataset used
was provided for the Kaggle TensorFlow speech recognition
challenge[2][3]. This dataset contains 10 labelled commands
which are “yes”, “no”, “up”, “down”, “left”, “right”, “on”, “off”,
“stop”, and “go”. In addition to these, the dataset is augmented with
two additional classes which are “unknown” and silence. The “unknown”
class contains other utterances that are not the commands that we are
trying to catagorize. The silence class corresponds to no
utterance, but rather background noise. The input data is provided as
audio clips in WAV format sampled at 16Khz. This data set therefore
maps an audio file to 12 possible classes. The audio provided is all
approximately 1 sec in duration, with some slight variation in
length. The dataset contains 64,727 audio samples. With the samples
are also provided lists of samples for both validation and test set
splitting. There are 6835 valiation samples and 6798 test samples
provided. The remaining of the 64,727 samples are used for training.

The entire data set comprises 30 possible spoken commands in
approximately equal distribution. Of these, only 10 are commands that
are to be differentiated. The remaining 20 are other words that are
spoken which get pooled into the \"unknown\" class and are not
differentiated.

The 64,727 audio samples were preprocessed. For each sample, if the
sample belonged to the 10 classes that we are interested in, they were
labeled with that class, otherwise they were labeled with the
\"unknown\" class. If the sample was background noise, it was labeled
with the \"silence\" class. After the samples were re-binned into the
classes that we are interested in, they were all converted to a common
size. Most samples are of 1 second in duration at 16Khz, or 16,000
samples in length. The regularity of the data allows for easily
creating batches of sequences for batch processing. Any audio clips
that are too long have a 16,000 sample window taken from the too long
length clip. Any audio clips that are too short are padded to the correct
length.



Describe your dataset: how many training/validation/test examples do you have? Is there
any preprocessing you did? What about normalization or data augmentation? What is the
resolution of your images? How is your time-series data discretized? Include a citation on
where you obtained your dataset from. Depending on available space, show some examples
from your dataset. You should also talk about the features you used. If you extracted
features using Fourier transforms, word2vec, PCA,
ICA, etc. make sure to talk about it. Try to include examples of your data in the report
(e.g. include an image, show a waveform, etc.).



\section{ Methods }
Describe your learning algorithms, proposed algorithm(s), or theoretical proof(s). Make
sure to include relevant mathematical notation. For example, you can include the loss function you are using. It is okay to use formulas from the lectures (online or in-class). For each algorithm, give a short description 
of how it works. Again, we are looking for your understanding of how these deep
learning algorithms work. Although the teaching staff probably know the algorithms, future
readers may not (reports will be posted on the class website). Additionally, if you are
using a niche or cutting-edge algorithm (anything else not covered in the class), you may want to explain your algorithm using 1/2
paragraphs. Note: Theory/algorithms projects may have an appendix showing extended
proofs (see Appendix section below).

\section{Experiments/Results/Discussion}
You should also give details about what (hyper)parameters you chose (e.g. why did you
use X learning rate for gradient descent, what was your mini-batch size and why) and how
you chose them. What your primary metrics are: accuracy, precision,
AUC, etc. Provide equations for the metrics if necessary. For results, you want to have a
mixture of tables and plots. If you are solving a classification problem, you should include a
confusion matrix or AUC/AUPRC curves. Include performance metrics such as precision,
recall, and accuracy. For regression problems, state the average error. You should have
both quantitative and qualitative results. To reiterate, you must have both quantitative
and qualitative results! If it applies: include visualizations of results, heatmaps,
examples of where your algorithm failed and a discussion of why certain algorithms failed
or succeeded. In addition, explain whether you think you have overfit to your training set
and what, if anything, you did to mitigate that. Make sure to discuss the figures/tables in
your main text throughout this section. Your plots should include legends, axis labels, and
have font sizes that are legible when printed.

\section{Conclusion/Future Work }
Summarize your report and reiterate key points. Which algorithms were the highestperforming?
Why do you think that some algorithms worked better than others? For
future work, if you had more time, more team members, or more computational resources,
what would you explore?

\section{Contributions}
The contributions section is not included in the 5 page limit. This section should describe
what each team member worked on and contributed to the project.

\section*{References}
This section should include citations for: (1) Any papers mentioned in the related work
section. (2) Papers describing algorithms that you used which were not covered in class.
(3) Code or libraries you downloaded and used. This includes libraries such as scikit-learn, Tensorflow, Pytorch, Keras etc. Acceptable formats include: MLA, APA, IEEE. If you
do not use one of these formats, each reference entry must include the following (preferably
in this order): author(s), title, conference/journal, publisher, year. If you are using TeX,
you can use any bibliography format which includes the items mentioned above. We are excluding
the references section from the page limit to encourage students to perform a thorough
literature review/related work section without being space-penalized if they include more
references. Any choice of citation style is acceptable
as long as you are consistent. 

\medskip
\small
[1] Alexander, J.A.\ \& Mozer, M.C.\ (1995) Template-based algorithms
for connectionist rule extraction. In G.\ Tesauro, D.S.\ Touretzky and
T.K.\ Leen (eds.), {\it Advances in Neural Information Processing
  Systems 7}, pp.\ 609--616. Cambridge, MA: MIT Press.

[2] Bower, J.M.\ \& Beeman, D.\ (1995) {\it The Book of GENESIS:
  Exploring Realistic Neural Models with the GEneral NEural SImulation
  System.}  New York: TELOS/Springer--Verlag.

[3] Hasselmo, M.E., Schnell, E.\ \& Barkai, E.\ (1995) Dynamics of
learning and recall at excitatory recurrent synapses and cholinergic
modulation in rat hippocampal region CA3. {\it Journal of
  Neuroscience} {\bf 15}(7):5249-5262.

\end{document}
